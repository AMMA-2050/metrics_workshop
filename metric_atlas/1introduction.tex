\section{Introduction} \label{sec:intro}

This atlas has been created to provide relevant and up-to-date information on projected climate change in West Africa by the 2050s, crucially incorporating information on the modelling uncertainties that still persist. It forms a series of atlases, each addressing a specific region, month or season.

These atlases have been created by a team of African and European climate scientists within the FCFA AMMA-2050 project. The project aims to improve understanding of how the West African monsoon will be affected by climate change in the coming decades and to enhance the capacity of West African societies to prepare and adapt. 

\textbf{What is a Climate Metric?} Our aim is to present maps and graphs of the changes and uncertainties in climate metrics that are relevant to stakeholders. We use the term 'climate metric' to describe a statistical measure of an aspect of climate that may change in the future, and that is thought to be relevant for assessing the effects of high impact weather and climate in West Africa. These climate metrics were chosen by groups of AMMA-2050 impacts scientists, in consultation with the project's climate scientists. We have also taken some feedback from stakeholders.

\textbf{Audience:} The audience for this atlas is envisaged to be:

\begin{itemize}
  \item \textbf{\textit{Climate change impact scientists}}, principally from the hydrological and agricultural communities. We anticipate that suitably presented information on the changes and uncertainties in the most relevant aspects of West African climate can help interpret the output from impacts models, such as changes in crop yields, flooding frequency, etc.

  \item \textbf{\textit{Technical experts}} in government ministries and local, national, regional and international bodies, engaged in sectors and services directly impacted by climate variability and change, and who have some understanding of modelling climate change. This group contributes to informed resource management decisions based on the plausible range of future climate outcomes. They will likely benefit from an understanding of the climate change context behind the outputs of impact modelling that AMMA-2050 will provide, such as changes in crop yields and flooding frequency. 

\end{itemize}

\textbf{Context of AMMA-2050 Stakeholder Communication:} A key outcome of AMMA-2050 is the development of appropriate communication tools to help stakeholders understand the predicted impacts and uncertainties for each of their sectors. This atlas is not seen as one of these primary communication tools, but rather as an optional supplementary technical source of information that those stakeholders who have some understanding of modelling climate change may choose to use.

\textbf{Data:} Climate model data was sourced from the CMIP5 archive (which is also that used to provide climate model projections for the IPCC 5th Assessment Report), and then post-processed at the Institut Pierre Simon Laplace (IPSL) to disaggregate the projections to a 0.5\textdegree grid (approx 50x50km) (a process sometimes called ‘downscaling’), and to largely eliminate discrepancies between historical model simulations and observations (a process called 'bias correction'). Further details of these techniques are being prepared for publication (updates will be available at \url{http://www.amma2050.org/content/publications}). The main focus is on future projections forced by 'RCP8.5' which is a high-end anthropogenic emissions scenario. This roughly matches current emissions, and can be linked to the worst plausible outcome, but note however that compliance with the Paris accord would lead to smaller changes in climate than those shown here. Indeed a minority of figures also show a comparison with strong mitigation scenarios (RCP2.6 and RCP4.5). Data from 29 climate models are used for the RCP8.5 scenario, 27 for RCP4.5, and 20 for RCP2.6.

\textbf{Sub-Atlases:} This atlas is one in a series of atlases. These cover combinations of 5 regions and up to 7 months (May to November) and the northern wet season (defined as July to September. The regions are defined as the land points within: Senegal, Burkina Faso, Sahelian (12.5\textdegree to 17.5\textdegreeN, 11\textdegreeW to 30\textdegreeE), Soudanian (9.5N\textdegree to 12.5\textdegreeN, coast to 30\textdegreeE) and Guinea Coast (9.5N\textdegree to 12.5\textdegreeN, coast to 10\textdegreeE).

\textbf{Metrics:} The climate metrics analysed here cover the majority of those included within the AMMA-2050 Technical report No. 1 (\url{http://www.amma2050.org/content/technical-reports}). Exceptions are evapotranspiration and humidity-based metrics (not yet bias-corrected and disaggregated), wet season duration (awaiting a definition), seasonal mean temperature from onset date (over-sensitive to onset definition and coding complexities), and 4 medium-priority metrics (drought severity, max seasonal no. of consecutive dry days, diurnal temperature range, and count of extreme Tmax days; all due to coding complexities).

\textbf{Plot Types:} We deliberately choose to display the changes and uncertainties in model projections using a number of different formats. Maps are used to show spatial differences in the projected changes and their uncertainties, using the 90\textsuperscript{th} and 10\textsuperscript{th} percentiles computed across the multi-model ensemble to illustrate the range of plausible outcomes. To be clear, these percentiles are computed across model space (usually 29 models), not for example across different days within a model. Note also that we deliberately omit the median response to encourage planning for a range of plausible outcomes rather than only the central outcome. The mapped figures also include the recent climatology of each metric. Then, for each metric, we use a histogram, box-plots and a ranked scatter-plot to further illustrate the uncertainty (across the ensemble of climate models) in the projected change of that metric averaged across the entire region. The box-plots also include analysis of scenarios with stronger mitigation of anthropogenic emissions. Additionally, for those metrics that are defined as a monthly average, the annual cycle of their changes and uncertainties are shown using box-plots. Last, the units plotted are either those of the relevant metric or (for precipitation-based metrics) the percentage difference from their climatology.

\textbf{Contacts to Request Data or Software:}
Bias-corrected 0.5\textdegree CMIP5 Data: S. Janicot \href{mailto:serge.janicot@locean-ipsl.upmc.fr}{serge.janicot@locean-ipsl.upmc.fr}
Climate Metrics Data: C. Klein \href{mailto:cornkle@ceh.ac.uk}{cornkle@ceh.ac.uk} or R. Fitzpatrick \href{mailto:js08rgjf@leeds.ac.uk}{js08rgjf@leeds.ac.uk}
Climate Metrics Software:
\url{https://github.com/AMMA-2050/metrics_workshop/tree/master/metric_atlas}
