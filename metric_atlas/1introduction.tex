\section{Introduction} \label{sec:intro}

This atlas has been created to provide relevant and up-to-date information on projected climate change in West Africa by the 2050s, crucially incorporating information on the modelling uncertainties that still persist. It forms a series of atlases, each addressing a specific region, month or season, and dataset version.

These atlases have been created by a team of African and European climate scientists within the FCFA AMMA-2050 project. The project aims to improve understanding of how the West African monsoon will be affected by climate change in the coming decades and to enhance the capacity of West African societies to prepare and adapt. 

Our aim is to present maps and graphs of the changes and uncertainties in climate metrics that are relevant to stakeholders. We use the term 'climate metric' to describe a statistical measure of an aspect of climate that may change in the future, and that is thought to be relevant for assessing the effects of high impact weather and climate in West Africa. These climate metrics were chosen by groups of AMMA-2050 impacts scientists, in consultation with the project's climate scientists. We have also taken some feedback from stakeholders.

The audience for this atlas is envisaged to be:

\begin{itemize}
  \item \textbf{\textit{Climate change impact scientists}}, principally from the hydrological and agricultural communities. Suitably presented information on the changes and uncertainties in the most relevant aspects of West African climate should help interpret the output of impacts models, such as changes in crop yields, flooding frequency, etc.
  \item \textbf{\textit{Technical experts}} in government ministries and local, national, regional and international bodies, engaged in sectors and services directly impacted by climate variability and change. This contributes to informed resource management decisions based on the plausible range of future climate outcomes. They will likely benefit from an understanding of the climate change context behind the outputs of impact modelling that AMMA-2050 will provide, such as changes in crop yields and flooding frequency. A key outcome of AMMA-2050 is the development of appropriate communication tools to help stakeholders understand the predicted impacts and uncertainties for each of their sectors. This atlas is not seen as one of these primary communication tools, but rather as an optional supplementary technical source of information that those stakeholders who have some understanding of modelling climate change may choose to use.
\end{itemize}

Climate model data was sourced from the CMIP5 archive (which is also that used in the IPCC 5th Assessment Report), and then post-processed at IPSL to disaggregate the projections to a 0.5\textdegree grid (a process sometimes called 'downscaling'). Two versions are available (used in separate atlases), with and without bias correction, a process by which discrepancies between the historical model simulations and observations are corrected. Further details of these techniques are being prepared for publication.

The climate metrics analysed here are the majority of those published in AMMA-2050's Technical report No. 1. Exceptions are humidity-based metrics (not yet bias-corrected and disaggregated), wet season duration (awaiting a definition), seasonal mean temperature from onset date (over-sensitive to onset definition + coding complexities), and medium-priority metrics (drought severity, max seasonal no. of consecutive dry days, diurnal temperature range, extreme Tmax days ; all due to coding complexities). It is also clear that the simulation of a small minority of the metrics shown are not credible in large-scale climate models (for example the simulation of dry spells), but we nevertheless retain them to potentially aid the interpretation of the output of impacts models forced by this data. The number of wind events greater than 70km/hr could not be produced because this threshold is not reached at the scales represented by the models.
