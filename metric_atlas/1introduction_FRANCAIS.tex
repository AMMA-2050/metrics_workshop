\section{Introduction} \label{sec:intro}

Cet atlas a été créé pour fournir des informations pertinentes et actualisées sur les changements climatiques prévus en Afrique de l'Ouest d'ici les années 2050, en intégrant de manière cruciale des informations sur les incertitudes de modélisation qui persistent. Il forme une série d'atlas, chacun s'adressant à une région, un mois ou une saison spécifique.

Ces atlas ont été créés par une équipe de climatologues africains et européens dans le cadre du projet AMMA-2050, dans le cadre du programme Future Climate for Africa. Le projet vise à améliorer la compréhension de la façon dont l'Afrique de l'Ouest sera affectée par le changement climatique dans les décennies à venir et à évaluer comment cette meilleure compréhension peut soutenir un développement compatible avec le climat dans une prise de décision à moyen terme (5-40 ans).

\vspace{5mm}

\textbf{Qu'est-ce qu'une mesure climatique (Climate Metric)?}  Notre objectif est de présenter des cartes et des graphiques des changements et des incertitudes dans les mesures climatiques qui sont pertinentes pour les parties prenantes. Nous utilisons le terme «mesure du climat» pour décrire une mesure statistique d'un aspect du climat qui pourrait changer à l'avenir, et qui serait pertinent pour évaluer les effets du temps et du climat à fort impact en Afrique de l'Ouest. Ces mesures climatiques ont été choisies par des groupes de chercheurs sur les impacts AMMA-2050, en consultation avec les climatologues du projet. Nous avons également pris en compte les commentaires des parties prenantes.

\vspace{5mm}

\textbf{Audience:} L'audience de cet atlas est prévu d'être:

\begin{itemize}
  \item \textbf{\textit{Les scientifiques du changement climatique impact}}, principalement des communautés hydrologiques et agricoles. Nous anticipons qu'une information correctement présentée sur les changements et les incertitudes dans les aspects les plus pertinents du climat ouest-africain peut aider à interpréter les résultats des modèles d'impacts, tels que les changements dans les rendements des cultures, la fréquence des inondations, etc.

  \item \textbf{\textit{Les experts techniques}} des ministères et des organismes locaux, nationaux, régionaux et internationaux, engagés dans des secteurs et des services directement touchés par la variabilité et le changement climatiques, et qui ont une certaine compréhension de la modélisation du changement climatique. Ce groupe contribue à la prise de décisions éclairées en matière de gestion des ressources en fonction de la gamme plausible de résultats climatiques futurs. Ils bénéficieront probablement d'une compréhension du contexte du changement climatique qui sous-tend les résultats de la modélisation de l'impact qu'AMMA-2050 fournira, tels que les changements dans les rendements des cultures et la fréquence des inondations. 

\end{itemize}

\textbf{Contexte de la communication avec les parties prenantes AMMA-2050:} Un résultat clé de AMMA-2050 est le développement d'outils de communication appropriés pour aider les parties prenantes à comprendre les impacts et les incertitudes prévus pour chacun de leurs secteurs. Cet atlas n'est pas considéré comme l'un de ces outils de communication principaux, mais plutôt comme une source d'information technique complémentaire facultative que les parties prenantes qui ont une certaine compréhension de la modélisation du changement climatique peuvent choisir d'utiliser.

\vspace{5mm}

\textbf{Données:} les données du modèle climatique proviennent des archives CMIP5 (qui servent également à fournir des projections de modèles climatiques pour le 5ème rapport d'évaluation du GIEC), puis sont post-traitées à l'Institut Pierre Simon Laplace (IPSL) pour désagréger les projections en réseau de 0,5 \textdegree (environ 50x50km) (un processus parfois appelé «downscaling»), et d'éliminer en grande partie les écarts entre les simulations de modèles historiques et les observations (un processus appelé «correction de biais»). D'autres détails de ces techniques sont en cours de préparation pour publication (mises à jour seront disponibles à http://www.amma2050.org/content/publications). L'accent est mis sur les projections futures imposées par «RCP8.5» qui est un scénario d'émissions anthropiques haut de gamme. Cela correspond à peu près aux émissions actuelles, et peut être lié au pire résultat plausible, mais notez toutefois que le respect de l'accord de Paris conduirait à des changements climatiques moins importants que ceux présentés ici. En effet, une minorité de chiffres montre également une comparaison avec de forts scénarios d'atténuation (RCP2.6 et RCP4.5). Les données de 29 modèles climatiques sont utilisées pour le scénario RCP8.5, 27 pour RCP4.5 et 20 pour RCP2.6.

\vspace{5mm}

\textbf{Sous-Atlas:} Cet atlas fait partie d'une série d'atlas. Elles couvrent des combinaisons de 5 régions et jusqu'à 7 mois (de mai à novembre) et la saison des pluies du nord (définies comme juillet à septembre), les régions étant définies comme les points terrestres: Sénégal, Burkina Faso, Sahélien (12,5 \textdegree N à 17,5 \textdegree N, 11 \textdegree W à 30 \textdegree E), Soudanais (9,5 \textdegree N à 12,5 \textdegree N, côte à 30 \textdegree E) et côte de Guinée (9,5 \textdegree N à 12,5 \textdegree N, côte à 10 \textdegree E).

\vspace{5mm}
       
\textbf{Métriques:} Les mesures climatiques analysées ici couvrent la majorité de celles incluses dans le rapport technique AMMA-2050 n ° 1 (http://www.amma2050.org/content/technical-reports). Les exceptions sont l'évapotranspiration et les mesures basées sur l'humidité (pas encore corrigées en biais et désagrégées), la durée de la saison des pluies (en attente de définition), la température moyenne saisonnière à partir de date de début (plus sensible à la définition de début + complexité de codage), et 4 priorité moyenne (sévérité de la sécheresse, nombre maximal saisonnier de jours secs consécutifs, plage de température diurne et nombre de jours Tmax extrêmes, le tout en raison de complexités de codage).

\vspace{5mm}

\textbf{Types de graphique:} Nous choisissons délibérément d'afficher les changements et les incertitudes dans les projections de modèles en utilisant un certain nombre de formats différents. Les cartes sont utilisées pour montrer les différences spatiales dans les changements projetés et leurs incertitudes, en utilisant les 90e et 10e centiles calculés à travers l'ensemble multi-modèle pour illustrer la gamme des résultats plausibles. Pour être clair, ces percentiles sont calculés à travers l'espace du modèle (habituellement 29 modèles), et non par exemple à travers différents jours dans un modèle. Notez également que nous omettons délibérément la réponse médiane pour encourager la planification d'une gamme de résultats plausibles plutôt que seulement le résultat central. Les figures cartographiées incluent également la climatologie récente de chaque métrique. Ensuite, pour chaque métrique, nous utilisons un histogramme, des diagrammes en bâtons et un diagramme de dispersion classé pour illustrer davantage l'incertitude (à travers l'ensemble des modèles climatiques) dans le changement projeté de cette métrique dans l'ensemble de la région. La boîte-parcelles  comprennent également l'analyse des scénarios d'atténuation plus forte des émissions anthropiques. De plus, pour les mesures qui sont définies comme une moyenne mensuelle, le cycle annuel de leurs changements et de leurs incertitudes est indiqué à l'aide de diagrammes. Enfin, les unités tracées sont soit celles de la métrique pertinente, soit (pour les mesures basées sur les précipitations) la différence en pourcentage de leur climatologie.

\vspace{5mm}

\textbf{Contacts pour demander des données ou des logiciels:}
Données CMIP5 corrigées de biais de 0,5\textdegree: S. Janicot \href{mailto:serge.janicot@locean-ipsl.upmc.fr}{serge.janicot@locean-ipsl.upmc.fr} \newline
Données sur les mesures climatiques: C. Klein \href{mailto:cornkle@ceh.ac.uk}{cornkle@ceh.ac.uk} ou R. Fitzpatrick \href{mailto:js08rgjf@leeds.ac.uk}{js08rgjf@leeds.ac.uk} \newline
Logiciel Climate Metrics: \url{https://github.com/AMMA-2050/metrics_workshop/tree/master/metric_atlas} \newline

